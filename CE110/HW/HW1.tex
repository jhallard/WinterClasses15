\documentclass[a4paper,11pt]{article}
\usepackage{amsmath}
\usepackage{wrapfig}
\usepackage{fancyhdr}
\usepackage{graphicx}
\usepackage{url}
\usepackage{float}
\usepackage{amsmath}
\usepackage{amssymb}
\usepackage[margin=0.8in]{geometry}

% \setlength{\voffset}{-0.5in}
\setlength{\headsep}{5pt}
\newcommand{\suchthat}{\;\ifnum\currentgrouptype=16 \middle\fi|\;}
\newcommand{\answer}{\textbf{Answer : }}
\newcommand{\cd}{\texttt}
\newcommand{\benu}{\begin{enumerate}}
\newcommand{\enu}{\end{enumerate}}


%===========---------================
% Author John H Allard
% CMPE 12, HW #4
% December 10th, 2014
%===========---------================

\begin{document}
   % \vspace*{\stretch{-0.5}}
   \begin{center}
      \Large\textbf{CMPE 110 Homework \#1}\\
      \large\texttt{John Allard} \\
      \small\texttt{November 20th, 2014} \\
      \small\texttt{ID : 1437547}
   \end{center}
   % \vspace*{\stretch{0.5}}

%==========================================
%==========================================
%====== Begin Problems, 15 Total ==========
%==========================================
%==========================================

\benu
% ======== PROBLEM #1 ======== %
\item Question \#1 - Power \\

\benu
\item Question \#1A : `...However, we discussed that nowadays power density and heat has become an issue preventing scaling of frequency. Discuss why power and temperature are becoming an issue'.
\answer  \\

\item Question \#1B : Given the rough formulations governing power and frequency for each voltage region, discuss which region (consider Near and Super-threshold only) is more energy efficient.
\answer \\
To start, we will make the following assumptions.
% $$\text{Given } k \in \mathbb{R}^+ : $$ 
 $$\text{Near-threshold voltage } = V_{nth} = k \text{, } \text{Super-threshold voltage } = V_{sth} = 2k : k \in \mathbb{R}^+ $$
 $$ \text{Power} \propto V^3 \text{, Delay } \propto \frac{1}{V} \text{ , Energy } \propto \text{ Power } \times \text{ Delay} $$
 $$ \text{Energy Efficiency } \propto \text{ Energy } \times \text{ Delay } = \text{ Power } \times \text{ Delay}^2 $$

 The energy efficiency can then be calculated for both the near and super threshold voltage levels, as shown below. 

 \begin{tabular}{l || r}
 Near-Threshold  & Super-Threshold \\ \hline \\
 $P_{nth} = V_{nth}^3 = k^3$ & $P_sth = V_{sth}^3 = (2k)^3 = 8k$ \\
 $D_{nth} = \frac{1}{V_{nth}} = \frac{1}{k}$ & $D_{sth} = \frac{1}{2k}$ \\ 
 $E_{nth} = P_{nth}*D_{nth} = \frac{k^3}{k} = {k}^2$ & $E_{sth} = P_{sth}*D_{sth} = \frac{8k^3}{2k} = {4k}^2$ \\
 $EE_{nth} = E_{nth}*D_{nth} = \frac{k^2}{k} = k$ &  $EE_{sth} = E_{sth}*D_{sth} = \frac{4k^2}{2k} = 2k$ \\
 \end{tabular}
 
 Thus voltage levels that are near the threshold are more energy efficient. \\

\enu


\item Question \#2 Computing ISA's

 \benu 
 \item Question \#2A x86 CISC ISA \\ Fill out the first row of the above table (from the handout), but assume 32-bit data values. \\
 \answer 


 \item 

 \enu


\item Provide the output for each of the following code statements. \\


\item For each of the following items, identify whether the caller fucntion or the callee function performs the actions.

\item TODO

\item Write a C program that computes the pig-latin translation of an english word. \\


\enu

\end{document}